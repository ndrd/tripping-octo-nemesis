\documentclass[12pt]{article}
%\documentclass[12pt]{scrartcl}
%\setkomafont{disposition}{\normalfont\bfseries}
% pre\'ambulo

\usepackage{lmodern}
\usepackage[T1]{fontenc}
\usepackage[utf8]{inputenc} %Codificacion utf-8

\usepackage{mathtools}
\usepackage{graphicx}
\graphicspath{ {/home/socrates/projects/computer_vision/visual-cosmic-rainbown/testImages} }

\title{Visi\'on por Computadora  \\Pr\'actica 1 (reporte)}
%\subtitle{Pr\'actica}
\author{Jonathan de Jes\'us Andrade L\'opez}
\date{10 de marzo de 2015 \\Facultad de Ciencias UNAM}

\begin{document}
% cuerpo del documento
\maketitle
Detalles de la implementaci\'on, problemas y momentos de diversion\\
en la pr\'actica 1.

\begin{description}
  \item[Objetivo:] 
  Mejorar la intensidad de color y luminiscencia, adem\'as de aplicar fltrado
  en im\'agenes mediante correlaci\'on y convoluci\'on
  
  \item[Introducci\'on:]
  Un punto clave de este curso es la extracci\'on de informaci\'on de una im\'agen,
  para ello, es preciso tratarla antes de sacarle el mayor provecho a nuestros pixeles.
  Parte de ese proceso de tratamiento incluye mejorar su color, cambiar sus dimensiones, usar otro sistema de coordenadas,
  suavizarla, para mejorar las car\'acteristicas que realmente nos pueden servir a procesar
  correctamente la im\'agen. Estos filtros o procedimientos son b\'asicos, forman parte 
  de algoritmos m\'as complejos que posteriormente revisaremos.
  
  Los filtros de \textbf{Desenfoque gaussiano} y \textbf{Laplaciano del gaussiano}, se calculan
  usando matrices de convoluci\'on, i.e iterando sobre los pixeles de la imagen, tomaremos
  una matriz con un tama$ñ$o en relaci\'on al valor de la intensidad con la que queremos aplicar el filtro.
  Esta matriz, contiene los \textit{pesos} relativos a la vecindad del pixel, y asi ponderar su importancia
  relativa al pixel angular del filtro en ese indice de la iteraci\'on.
  
  El resto de los filtros son lineales, pues solo se recorre la im\'agen una unica vez y esto basta
  para obtener el resultado requerido.
  
  \item[Desarrollo:] 
  Los m\'ultiples filtros que se implementaron en esta pr\'actica son base para mejorar las condiciones con
  las que recibimos una im\'agen, pueden ayudar a mejorar su color, brillo, contraste, suavizarla, en otras 
  palabras, preparla para la extracci\'on de informaci\'on y manipulaci\'on m\'as espec\'ifica dependiendo 
  del prop\'osito de nuestro procesamiento.
  
  \begin{itemize}
     \item \textbf{Balance de color:} Se usa para mejorar la intensidad del color de las im\'agenes en base 
     al muestreo de intensidades de color de cada canal en el espectro RGB. Tomando el valor m\'aximo y el m\'inimo
     en el intervalo [0, 255], y haciendo la regla de correspondencia de estos valores hacia el cero absoluto y el m\'aximo
     absoluto dentro del intervalo. En im\'agenes con una gama baja y constante de colores, el filtro tiene resultados 
     visibles, pero, cuando al menos un pixel esta en alguna de los l\'imites del intervalo, su efecto es nulo. 
     Como soluci\'on a este problema, en lugar de tomar los valores extremos del espectro de intensidades, se puede tomar la
     media de cada canal, y hacer un proceso similar en dos direcciones, de la media hacia el cero, y de la media hacia el 255. Sus resultados son m\'as visibles cuando se aplica despu\'es de aumentar el brillo del canal de la im\'agen (desplazarla en la escala)
     
     \item \textbf{Blending:} Este filtro es de los m\'as sencillos, pues requiere de dos im\'agenes mezclando los
     pixeles de su \'area de intersecci\'on bajo la siguiente regla de correspondencia:
     
   \begin{center}
       \textit{ C = (1 - $\alpha$)B + $\alpha$F}
    \end{center}
     Con B y F las im\'agenes de fondo y de primer plano respectivamente.
     %\includegraphics{blendingHO}
     
    \item \textbf{Ajuste de contraste:} Este filtro tiene mayor presencia en la im\'agen despu\'es de aplicarse,
    mejora el ajuste general de la imagen, pues no solo hace una correspondencia lineal entre las magnitudes de
    de los colores de cada canal, si no que toma en cuenta la \textit{funci\'on de probabilidad acumulada} es decir
    distribuye uniformemente la intensidad de los colores mapeando las tonalidades anteriores a unas nuevas en 
    funci\'on de su distribuci\'on.
    
    \item \textbf{Desenfoque Gaussiano:} Uno de los filtros m\'as importantes dentro del procesamiento de im\'agenes,
    pues permite que cada pixel tome "informaci\'on" de sus vecinos, ponderando su imporntacia en funci\'on de su
    distancia, es decir, los vecinos m\'as cercanos, tendr\'an m\'as peso al momento de dejar su parte en el pixel
    de atenci\'on. Aplicar este filtro, requiere de una matriz de convoluci\'on, es decir, se iterara sobre las
    dimensiones de la im\'agen, pero adem\'as se recorrera linealmente la matriz (de tamaño 2$\sigma$ + 1) para
    poder pesar sus pixeles vecinos. Una ventaja de este algoritmo es que su kernel (o matriz de convoluci\'on)
    es separable, es decir, se puede aplicar primero en sentido horizontal, y despu\'es en vertical obteniendo 
    el mismo resultado, esta propiedad se debe a que a funci\'on gaussiana es simetrica y radial. Esta propiedad
    es preciosa en sentido computacional, pues reduce la complejidad del algoritmo.
    \begin{center}
     \textit{O(wh4$\sigma$$^{2}$) $\longrightarrow$ O(wh4$\sigma$) }
     \end{center} 
     Donde \textit{w, h} son el ancho y alto de la imagen, y \textit{$\sigma$} el tamaño del kernel, en 
     t\'erminos de tiempo, con una imagen de 12 Megap\'ixeles, el tiempo se reduce de \textit{ 40.95w ms}
     a tan solo \textit{ 9.8 ms} (Java 8 64 bit GNU/Linux)
     
     \item \textbf{Laplaciono del Gaussiano (Aproximaci\'on): } Como vimos en clase, el operador Laplaciano es la regla de 
     la cadena de doble derivada de la funcion gaussiana, viendo el resultado geom\'etrico, podemos notar que 
     su resultado es muy similar al de aplicar el filtro Gaussiano a la im\'agen y restar estos valores a sus
     intensidades originales, de esta manera, obtenemos solo aquellos colores de gama alta, dado que s\'olo es una
     aproximaci\'on, el resultado es bueno, pero no el m\'as preciso, pero por fines pragm\'aticos, este es el 
     m\'etodo m\'as utilizado.
     
     \item \textbf{Im\'agenes Hibridas: } En este filtro usamos los dos anteriores, primero, tomamos una im\'agen, 
     aplicamos el desenfoque Gaussiano para eliminar las frecuencias m\'as altas, despu\'es, tomamos una segunda 
     im\'agen y a esta le aplicacimos el laplaciano, de esta manera nos quedamos las frecuencias m\'as altas de 
     una y las m\'as altas de la otra, dando un efecto visual, que se percibe  a diferentes distancias. Cabe 
     mencionar que el valor de la $\sigma$ es independiente para cada filtro, de esta manera, obtendremos el 
     resultado deseado, y como es evidente, la forma de logra el efecto, es mezclando las imagenes usando el 
     blending en una proporci\'on de \textit{ $\alpha$ = 0.5}\\
     
     \item \textbf{Mundo pequeño: } En cuestion de implementaci\'on, este fue el filtro que m\'as tarde en
     solucionar. Consiste en tomar una imagen panoramica, rotarla, hacerla cuadrada y por \'ultimo, aplicar
     una transformacion polar para verla como un c\'irculo. Uno de los problemas que m\'as tarde en resolver 
     fue el de mapear la imagen cartesiana a formato polar, primero intente iterar sobre un c\'irculo sobre 
     el di\'ametro y el valor de su radio, pero como consecuencia, muchos de los puntos no estaban definidos
     bajo esta funci\'on, as\'i que como nos habian mencionado en clase, el truco estaba en iterar sobre la 
     im\'agen resultado, calculando a cada pixel el valor de \textit{r} y de \textit{$\Theta$}, una vez que los
     tenemos, solo basta considerar, que el ancho de la imagen equivale a 2$\pi$ y el alto al valor esta en 
     relaci\'on con \textit{r} y as\'i obtenemos el relleno total de los pixeles destino de la im\'agen.
     
     \item \textbf{Interpolaci\'on} Sirve para poder escalar (UpScale, DownScale) una im\'agen, usando
     la formula matem\'atica de la interpolac\'ion bilienal, usando los pixeles aledaños para calcular
     el valor del pixel, que esta entre ellos (en caso de que se se este aumentado el tamaño). Es u proceso
     relativamente r\'apido y con buenos resultados, aunque en mi caso, aun tengo problemas con algunos colores.
     
      
   \end{itemize} 
   
   \item[Conclusi\'on:]
   Los diferentes filtros que se implementaron en esta pr\'actica, son lo fundamental para poder 
   desarrollar procedimientos m\'as complejos con el tratamiento de im\'agenes, el obejtivo fundamental
   de tratar las im\'agenes es dejarlas en las mejores condiciones para extraer de ellas las car\'acteristicas
   que nos pueden dar m\'as informaci\'on de la im\'agen. Mejorar el color, el contraste, cambiar el 
   tamaño, suavizar y cambiar el sistema de coordenadas, pueden ser de gran utilidad para futuras aplicaciones
   dentro del curso, as\'i que por esta pr\'actica, considero cumplido el objetivo.
  
\end{description}

\end{document}
